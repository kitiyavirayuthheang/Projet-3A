\documentclass[french]{beamer}
\usepackage[utf8]{inputenc}
\usepackage[T1]{fontenc}
\usepackage{graphicx}
\usepackage{caption}
\usepackage{babel}

% This is the file main.tex

\usetheme{Berlin}
\title{Modélisation et détection de délit d'initié \\ Soutenance mi-parcours P3A}
\author{Heang Kitiyavirayuth, Lucas Broux}
\date{\today}

% Create a frame at the beginning of a section.
\AtBeginSection[]{
  \begin{frame}
  \vfill
  \centering
  \begin{beamercolorbox}[sep=8pt,center,shadow=true,rounded=true]{title}
    \usebeamerfont{title}\insertsectionhead\par%
  \end{beamercolorbox}
  \vfill
  \end{frame}
}

% Document.
\begin{document}

\begin{frame}
\titlepage
\end{frame}

\section*{Outline}
\begin{frame}
\tableofcontents
\end{frame}


\section{Introduction}
\subsection{Sujet choisi, problématique générale}
\begin{frame}
  \begin{itemize}[<+->]
    \item \textbf{Modélisation et détection de délit d'initié.}
    \item Problématique concrète mais actuellement assez mal résolue.
\end{itemize}
\frametitle{Sujet choisi}
\end{frame}

\subsection{Objectifs du projet}
\begin{frame}
  \begin{itemize}[<+->]
    \item Comprendre et analyser des articles sur le sujet :
    \begin{itemize}
%    \item [1] A. Grorud, M. Pontier, Comment détecter le délit d'initié, C.R. Acad. Sci. Paris, t. 324, p. 1137-1142, 1997.
	\item [1] A. Grorud, M. Pontier, Insider trading in a continuous time market model, International Journal of Theoretical and Applied Finance, 1, p. 331-347, 1998.
	\item [2] H. Föllmer, P. Imkeller, Anticipation cancelled by a Girsanov transformation: a paradox on Wiener space, Ann. Inst. H. Poincaré Probab. Statist 29.4, 569-586, 1993.
	\end{itemize}
    \item Pour la suite du projet : à préciser :
    \begin{itemize}
    \item Réalisation de simulations numériques.
    \item Analyse théorique de cas particuliers.
    \item ...
    \end{itemize}
\end{itemize}
\frametitle{Objectifs du projet}
\end{frame}

\section{Compréhension actuelle du problème}
\subsection{Modélisation mathématique}
% Problème qui revient à l'étude du comportement lorsqu'on étend la filtration, ...
\subsection{Jeu d'hypothèses}
% Mettre en évidence le jeu d'hypothèses qui permettent d'appliquer les théorèmes voulus : elles sont techniques et on s'est accordé sur le fait de ne pas les étudier en détail pour le moment.
\subsection{Raisonnement}

\section{Simulations}
% Ou plutot, tentatives de simulation ...

\section{Conclusion}
\begin{frame}
\frametitle{Conclusion}
\end{frame}


\begin{frame}
\end{frame} % to enforce entries in the table of contents
\end{document}
